\documentclass[a4paper]{article}
\usepackage[T1]{fontenc}
\usepackage[utf8]{inputenc}
\usepackage[{italian, english}]{babel}
\usepackage{verbatimbox}
\usepackage{mathtools}
\usepackage{fancyvrb}

\newenvironment{qanda}{\setlength{\parindent}{0pt}}{\bigskip}
\newcommand{\Q}{\bigskip\bfseries Q: }
\newcommand{\A}{\par\textbf{A:} \normalfont}


\begin{document}

\author{Fabio Chiarani}
\title{HELP - Linguaggi 2019}

\maketitle
\pagebreak

\section{Alcune cose...}
\textbf{Scope}: Lo scope di una variabile è il range di comandi nella quale è visibile. Le regole di scope determinano come i riferimenti ad un nome sono associati alle variabili.
\newline
Lo scope è \emph{statico} se il nome non locale è risolto nel blocco che lo racchiude.
\newline
Lo scope è \emph{dinamico} se + risolto nel blocco attivato più di recente, e NON ancora disattivato.
\newline
\newline
\textbf{Dichiarazioni}: sono una categoria sintattica nelle quali gli elementi sono elaborati per produrre legami.
\newline
\textbf{Espressioni}: sono una categoria sintattica dove le espressioni vengono valutate per ottenere valori esprimibili $Eval$.
\newline
\textbf{Comandi}: sono una categoria sintattica nella quale gli elementi sono eseguiti per aggiornare la memoria della macchina astratta che supporta il linguaggio.
\newline
\newline
\textbf{Passaggio per valore}: il parametro attuale viene utilizzato per inizializzare il valore del formare. Le modifiche del formale non passano all'attuale. Può essere implementato per copia.
\newline
\textbf{Passaggio per risultato}: nessun valore è trasmesso al sottoprogramma. Il valore del parametro formale viene copiato nell'attuale solo alla fine del sottoprogramma.
\newline
\textbf{Passaggio per valore-risultato}: sopra uniti
\newline
\textbf{Passaggio per riferimento}: passa per un cammino d'accesso. Non c'è copia o duplicazione, l'accesso è rallentato per l'indirizzamento indiretto. Crea alias che può causare side-effect.
\newline
\textbf{Passaggio per nome}: esegue la sostituzione testuale. I parametri fomrale sono legati al metodo d'accesso al momento della chiamata, i parametri attuali sono legati al valore durante il loro accesso. Permette il late binding e porta al conflitto dei nomi.

\pagebreak
\section{Domande e Risposte}
\begin{qanda}
	 
	\Q Cosa significa implementare un linguaggio \emph{L}? Definire esplicitamente e dettagliatamente tutti i concetti utilizzati.
	
	\A Un linguaggio di programmazione è un insieme di costrutti e regole per descrivere algoritmi e dati. 
	Implementare un linguaggio significa costrutire una macchina \emph{ML} che ne esegua i programmi $P \in PROG^L$.
	Questa macchina dovrà girare su una macchina ospite $M_0$ avente un linguaggio $L_0$. Ci sono due possibili approci: tramite interprete o tramite compilatore.
	\newline
	\newline
	\textbf{Interprete}: Il primo approcio è la traduzione implicita o \emph{interpretazione}. Un interprete $INT ^{L,L_0} : (PROG^L \times D) \to D$ è un programma
	scritto in $L_0$ tale che per ogni input $d \in D$ si ha: $INT(P,d) = P(d)$ per ogni $P \in PROG^L$
	\newline
	\newline
	\textbf{Compilatore}: Il secondo approcio è la traduzione esplicita o \emph{compilazione}. Traduco un programma scritto in $L$ in un programma scritto in $L_0$.
	Un compilatore è un programma che realizza la funzione $COMP^{L,L_0}:PROG^L \to PROG^L_0$ tale che per ogni input $d \in D$ si ha: 
	$COMP(P)(d) = P(d)$ per ogni $P \in PROG^L$
	\newline
	\newline
	\textbf{Confronto}: L'interprete è lento, ma è facile da implementare e è portabile. La compilazione rende l'esecuzione più veloce dell'interpretazione, ma è più difficile
	da realizzare per la distanza dei linguaggi. Inoltre si perde la struttura del programma originale.
	\newline
    \newline
    
\end{qanda}
\begin{qanda}
	\Q Descrivere intuitivamente cosa è una dichiarazione. Definire formalmente le dichiarazioni di IMP e dare semantica operazionale statica e dinamica alla dichiarazione di procedura.
	
	\A Le dichiarazioni sono una categoria sintattica i quali elementi sono elaborati per produrre associazioni tra un identificatore e un tipo denotabile.
	\newline
	\newline
	
\end{qanda}
\begin{qanda}
	\Q Definire cosa è un interprete: dare definizione semantica e descrivere la struttura.
	
	\A Un interprete per programmi di inguaggio $S$ permette di eseguire tali programmi su macchine astratte il cui linguaggio è $L$.
	Consideriamo due linguaggi turing completi $L,S$. E' garantita l'esistenza dell'interprete, cioè di un programma $int \in L$ tale che per ogni programma 
	$P \in S$ e per ogni input $d \in D$ si ha: $[[int]]^L(P,d) = [[P]]^S(d)$
	\newline
    \newline
    
\end{qanda}
\begin{qanda}
	\Q Definire cosa è il passaggio di parametri, dando in particolare la definizione di passaggio per valore e per riferimento.
	
	\A Il passaggio di parametri è il binding tra un parametro locale e il valore denotato ottenuto dalla valutazione dell'argomento attuale.
	\textbf{Passaggio per valore}: il valore del parametro attuale viene utilizzato per inizializzare il valore del formale. Le modifiche del formale non passano all'attuale.
	Può essere implementato tramite copia o cammino d'accesso.
	\textbf{Passaggio per riferimento}: passa in cammino d'accesso, non c'è copia o duplicazione, ma l'accesso è rallentato dall'indirizzamento indiretto. Crea alias che causano side-effects.
	\newline
    \newline
    
\end{qanda}
\begin{qanda}
	\Q Definire intuitivamente e formalmente il concetto di specializzazione e specializzatore.
	    
	\A La specializzazione è la valutazione parziale di un programma su un input fissato. Formalmente, se $L,S,T$ sono linguaggi turing completi sull'insieme di input $D$
	allora un programma $spec \in L$ è uno specializzatore se per ogni ...formula impossibile...
	\newline
    \newline
    
\end{qanda}
\begin{qanda}
	\Q Definire cosa è una macchina astratta: dare la struttura e la definizione di linguaggio macchina
	        
	\A Il supporto di un linguaggio è la macchina astratta. Una macchina astratta di un linguaggio $L$ è un insieme di strutture dati ed algoritmi per memorizare ede eseguire programmi $L$. Viene denotata con
	$M_L$. Data la macchina astratta $M$ definiamo $ML$ linguaggio macchina, il linguaggio che ha come elementi tutte le stringhe interpretabili da $M$.
	Una macchina astratta è formata da memoria che immagazzina dati e programmi, ed interprete che esegue le istruzioni dei programmi. Esistono dipologie di operazioni
	e modalità di esecuzione indipendenti dal linguaggio (elaborazione dati primitivi, controllo sequenza di esecuzione, controllo dei dati, controllo della memoria).
	\newline
    \newline
    
\end{qanda}
\begin{qanda}
	\Q Definire cosa sono le politiche di binding. Spiegare in particolare la differenza tra shallow e depep, facendo attenzione a come si combinano con lo scoping statico e/o dinamico.
	            
	\A Le politiche di binding intervengono solo quando una procedure è passata come parametro ad un altra procedura.
	Il binding è deep se vale l'ambiente alla creazione del legame $h \to f$, oppure shallow se vale l'ambiente alla chiamata di $f$ attraverso $h$. Con scoping statico si utilizza sempre il deep.
	\newline
    \newline
    
\end{qanda}
\begin{qanda}
	\Q Definire il concetto di ricorsione.
	                
	\A La ricorsione è un modo alternativo all'iterazione per ottenre il potere esperessivo di una $MdT$. Una funzione è ricorsiva se è definita in termini di se stessa.                  
\end{qanda}
    

\end{document}